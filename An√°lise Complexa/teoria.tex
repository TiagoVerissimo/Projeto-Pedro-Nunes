\documentclass{article}

\usepackage{graphicx} % Required for inserting images
% Essential AMS packages for mathematical writing
\usepackage{amsmath, amssymb, amsthm, mathtools, bm}
% Extra symbol packages
\usepackage{mathrsfs, esint, xfrac}
% Improved matrix environments
\usepackage{nicematrix}
% Useful for physics-like notation (derivatives, vectors, etc.)
\usepackage{physics}

% Graphics and diagrams
\usepackage{pgfplots}
\pgfplotsset{compat=1.16}
\usepackage{tikz, tikz-cd}
\usepackage[all]{xy}

% Customize page layout
\usepackage[a4paper, margin=1in]{geometry}

% List customization
\usepackage{enumitem}

% Cross-referencing and hyperlinks
\usepackage{hyperref}
\usepackage{cleveref}

% Colors for diagrams (optional)
\usepackage{xcolor}

% Set page background and text color
\usepackage{pagecolor} % For setting background color
\pagecolor{black} % Set background to black
\color{white} % Set font color to white

\usepackage{biblatex}
\addbibresource{references.bib}

% Custom theorem environments
\newtheorem{theorem}{Theorem}[section]
\newtheorem{lemma}[theorem]{Lemma}
\newtheorem{proposition}[theorem]{Proposition}
\newtheorem{corollary}[theorem]{Corollary}

% Definition and Remark environments
\theoremstyle{definition}
\newtheorem{definition}{Definition}[section]

\theoremstyle{remark}
\newtheorem{remark}{Remark}

\title{Introducão á Análise Complexa - Teoria}
\author{Equipa Krypton}

% ---------------Begining of the document------------------%


\begin{document}
% I will writing in english given that
% my keyboard has a english layout, after I will translate the work.


\maketitle

\begin{abstract}
    This text is a very complete coverage in Portuguese of the theory of Complex Analysis.
    One should go from not knowing nothing at all to be at research level.
\end{abstract}

\section{Pre-requisites}
To read this text the only thing the readers need is a good knowledge on real analysis in
one variable, we recommend for a graduate read in the subbject we recommend \cite{rudin_principles_1976}
and for an undergraduate approach we recommend \cite{abbott_understanding_2015}.

\section{Historical Context}
To be made after. I have first to know the specifics of the subject and only after
write the history of the subject.

\section{The Complex Number System}

\subsection{What are Complex Numbers?}

\begin{definition}[Complex Numbers]
    We define the complex numbers as the algebraic field $(\mathbb{R}^2,+,*)$ where
    $$
    \mathbb{R}^2 := \{(a,b) : a,b \in \mathbb{R}\}
    $$
    that is equiped with the following operations
    \begin{align*}
        & (a,b) + (c,d) = (a+b,c+d) \\
        & (a,b) * (c,d) = (ac-bd,bc+ad)
    \end{align*}
\end{definition}

\begin{remark}
    If we define a set
    $$
    \mathbb{C} := \{ a+bi : a,b \in \mathbb{R}\}
    $$
    where $i = \sqrt{-1}$ is called a "imaginary" number, and where the $+,*$ are defined as
    \begin{align*}
        & (a+bi) + (c+di) := a+c + i(b+d) \\
        & (a+bi)*(c+di) := ac-bd + i (ad + bc)
    \end{align*}
    we can actually make an isomorphism $\mathbb{R}^2 \cong \mathbb{C}$ via $(a,b) \mapsto a+bi$
    which is what is commonly called a complex number, there are good reasons to adopt this
    notation therefore that is what we will do from now onwards in the text.
\end{remark}

We said in the definition that  $\mathbb{C}$ was an algebraic field, therefore there must
exist multiplicative inverses, to find such objects we notice that for a complex number $z$
the following indentity holds 
$$
z^{-1} := \frac{a}{a^2 + b^2} - i (\frac{b}{a^2 + b^2})
$$

Just notice that $zz^{-1}=z^{-1}z = 1$ which is to say that indeed
we have a multiplicative inverse for any $z \ne 0$.

\begin{remark}
    We define the real part $\Re$ and the imaginary part $\Im$ of a complex number $z:=a+bi$ as
    the following $\Re z = a$ and $\Im z = b$.
\end{remark}

\begin{definition}[Absolute Value and Conjugate]
    Let $z$ be a complex number we define the absolute value $|z|$ of $z$ as the real number
    $|z|:= a^2 + b^2$ and the conjugate $\bar{z}$ of $z$ as the value $\bar{z}:= a-ib$.
\end{definition}

\begin{remark}
    The algebraic identity $z\bar{z}= |z|^2$ is good to know from times to times.
\end{remark}


\section{The Geometry of Complex Numbers}

% Zona bibliográfica
\newpage
\section*{Bibliography}
\printbibliography



\end{document}
